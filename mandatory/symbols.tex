% Define your symbols here. Do not use \chapter or \section here. If you want
% to section your list of symbols do ONLY use \subsection*{<section title>}.
% ---
% You can use the code fragment below and add/remove columns by you own needs.
% Currently, the table shows three columns: The first contains the symbol,
% the second the symbol's unit and the third the symbols name/description.
% The table's header will automatically appear in the document's main language.
% The table automatically spans over multiple pages.

% Use "siunitx" package for units

\begin{center}
	\begin{longtable}{p{0.3\textwidth}p{0.3\textwidth}p{0.3\textwidth}}
    	\endhead
        \textbf{\IfLanguageName{ngerman}{Größe}{Symbol}} & \textbf{\IfLanguageName{ngerman}{Einheit}{Unit}} & \textbf{\IfLanguageName{ngerman}{Beschreibung}{Description}}\\
        \midrule
        $F_\text{N}$ & \si{\newton} & Force\\
        \midrule
        $\tau$ & \si[inter-unit-product =\ensuremath{}]{\newton\meter} & Torque\\
        % Insert mathematic terms using the inline math environment $$ like in the example above.
        % Use \midrule to separate parts of your symbols table.
		% Enter \\ after every line.
    \end{longtable}
\end{center}