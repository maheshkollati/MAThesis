\chapter{Introduction}
\label{chap:introduction}

Modern commercial vehicles represent a quintessential example of cyber-physical systems, where sophisticated software enables precise control over complex mechanical components. The software controlling these vehicles has grown exponentially in complexity over recent decades, evolving from simple engine management to comprehensive control of virtually all vehicle functions. At the core of this evolution is the Electronic Control Unit (ECU)—a specialized computer that executes software to manage specific vehicle functions \cite{broy2006challenges}. Contemporary commercial vehicles contain dozens of interconnected \acp{ECU} working in concert to ensure optimal performance, efficiency, and safety across diverse operating conditions.

\section{Background and Context}
\label{sec:background}

The automotive industry has undergone a profound transformation over the past decades, evolving from predominantly mechanical systems to highly sophisticated mechatronic platforms \cite{pretschner2007software}. This evolution has been particularly pronounced in the commercial vehicle sector, where modern trucks rely on complex networks of Electronic Control Units to manage everything from engine performance to safety systems \cite{broy2006challenges}. These systems must adapt to a wide range of operational conditions, regulatory requirements, and market-specific configurations, creating a significant challenge in managing software variability.

At the heart of this variability management lies the Common Powertrain Controller (\ac{CPC})—a central \ac{ECU} managing critical functions related to engine and transmission control. The \ac{CPC}'s operation is governed by thousands of configurable parameters that determine how the powertrain behaves under specific conditions \cite{staron2021automotive}. These parameters influence everything from basic engine timing to sophisticated emission control strategies, making their precise configuration essential for vehicle performance, efficiency, and regulatory compliance.

The parameter management challenge is further complicated by the global nature of modern vehicle development. Commercial vehicles must conform to different emissions regulations, operate in diverse environmental conditions, and meet varying customer expectations across global markets. Consequently, a single vehicle model may require numerous parameter configurations, each tailored to specific combinations of market requirements, hardware configurations, and customer specifications \cite{trovao2024evolution}.

\section{Problem Statement}
\label{sec:problem}

The current approach to parameter management in commercial vehicle development relies predominantly on distributed Excel spreadsheets, a methodology that emerged during a period when parameter counts were manageable and development teams were smaller \cite{trovao2024evolution}. However, as software complexity has increased exponentially, this fragmented approach has introduced significant limitations and risks to the development process.

Development teams distributed across different locations must coordinate changes to thousands of parameters, track their versions, and ensure consistency across multiple vehicle platforms. The absence of a centralized version control system makes it exceptionally difficult to track changes effectively and manage releases. This situation becomes particularly critical when dealing with safety-critical parameters that directly influence vehicle performance and regulatory compliance.

The manual nature of current processes, combined with the lack of automated validation mechanisms, introduces substantial risks of data inconsistency, version conflicts, and delayed implementation of critical parameter updates. Parameter changes are not consistently verified against established rules and constraints, potentially leading to incompatible configurations or non-compliant behavior \cite{staron2021automotive}.

Integration with critical enterprise systems presents another significant challenge. The current process of synchronizing data with internal database systems involves several manual steps, consuming valuable development resources and introducing potential points of failure in the configuration management workflow. The absence of automated data validation and synchronization mechanisms creates additional risks for data integrity and consistency across these interconnected systems.

Furthermore, the increasing emphasis on rapid development cycles and continuous integration in the automotive industry demands a more sophisticated approach to parameter management \cite{broy2006challenges}. The existing system's limitations become particularly apparent when considering the need for simultaneous development of multiple vehicle variants, each requiring specific parameter configurations for different markets and regulatory environments.

These challenges collectively underscore the urgent need for a modern, database-driven solution that can address the complexities of contemporary automotive software development while providing a scalable foundation for future growth and adaptation.

\section{Research Objectives}
\label{sec:objectives}

This thesis aims to address the fundamental challenges in automotive parameter management through the development of database architecture for\ac{VMAP} (Variant Management and Parametrization), a web-based application for powertrain parameter configuration. The research objectives encompass both theoretical foundations and practical implementation considerations, focusing on creating a robust solution that meets the complex demands of modern vehicle development processes.

The primary research objective centers on developing a centralized database architecture that can effectively manage the complexity of powertrain parameters while maintaining data integrity and traceability \cite{williams2004web}. This architecture must support sophisticated version control mechanisms that can handle parameter variations across different development stages and vehicle variants. The system should provide comprehensive audit trails and change history, enabling development teams to track modifications and understand the evolution of parameter configurations over time.

A second crucial objective focuses on the implementation of a sophisticated version control system that addresses the unique requirements of parameter management in automotive software development. This system must go beyond traditional source code version control approaches to handle the complex relationships between parameters, their variants, and their applications across different vehicle platforms \cite{staron2021automotive}. The version control mechanism should support parallel development streams while maintaining consistency and preventing conflicts in parameter configurations.

The research also aims to establish a comprehensive role-based access control system that supports the diverse needs of different user groups within the development process. This includes creating specialized interfaces and permissions for Module Developers, Documentation Team members, Administrators, and Read-only Users, each with specific capabilities and restrictions aligned with their responsibilities \cite{sandhu1998role}. The access control system must balance security requirements with the need for efficient collaboration among development teams.

Integration with existing enterprise systems represents another critical objective of this research. The\ac{VMAP} system must establish seamless data exchange mechanisms with internal database systems, ensuring consistent information flow while minimizing manual intervention \cite{broy2006challenges}. This integration should support automated validation of parameter changes and provide mechanisms for maintaining data consistency across different systems.

A final key objective involves the development of database interfaces and query optimization strategies that will support the web-based interface implementation. While the actual User Interface (UI) development falls outside the scope of the thesis, the research will focus on designing efficient database structures, stored procedures, and APIs that enable seamless integration with the planned web interface \cite{pretschner2007software}. This includes developing optimized query patterns for complex operations such as parameter comparison, variant management, and release workflows, while ensuring robust data validation and business rule enforcement.

\section{Significance of the Study}
\label{sec:significance}

The significance of this research extends beyond addressing immediate technical challenges in parameter management. By developing a comprehensive database solution for variant management and parametrization, this work contributes to the broader field of automotive software engineering in several important ways.

First, the research advances the understanding of version control in parameter-centric systems, extending traditional concepts of software versioning to accommodate the unique characteristics of automotive parameter configurations. While considerable research has been conducted on code versioning, the versioning of parameter data presents distinct challenges that require specialized approaches \cite{bhattacherjee2015principles}. This thesis contributes to closing this gap by developing and evaluating new methods for parameter versioning in complex automotive systems.

Second, the work addresses critical industry needs for improved quality and efficiency in vehicle development. Commercial vehicle manufacturers face increasing pressure to reduce development time while managing growing software complexity and ensuring regulatory compliance across global markets \cite{broy2006challenges}. By providing a more robust and efficient parameter management solution, this research directly contributes to these industry priorities, potentially reducing development costs and improving vehicle quality through more consistent parameter configurations.

Third, the research advances the integration of database technology with domain-specific engineering processes. By developing specialized database structures and functions tailored to the unique requirements of automotive parameter management, this work demonstrates how database technology can be adapted to support complex engineering workflows \cite{elmasri2015fundamentals}. This integration perspective is valuable not only for automotive applications but also for other engineering domains facing similar challenges in managing complex, highly variable system configurations.

Finally, the research contributes to the growing field of model-based systems engineering by providing a structured approach to managing the parametric aspects of system models. As the automotive industry continues to adopt model-based approaches for system development, the management of parameter configurations becomes increasingly critical for maintaining model integrity and traceability \cite{staron2021automotive}. This thesis provides insights and solutions that support this evolution toward more systematic model-based development practices.

\section{Thesis Structure}
\label{sec:structure}

The thesis is organized into six chapters that systematically address the research objectives and present a comprehensive solution for automotive parameter management. The structure follows a logical progression from theoretical foundations through practical implementation, ensuring thorough coverage of both academic and industry perspectives.

Following this introduction, Chapter \ref{chap:theoretical-background} presents a comprehensive review of the theoretical background relevant to automotive electronic control systems and database management approaches. This chapter examines the hierarchical organization of automotive electronic systems, explores database management systems with a focus on relational databases, investigates database design methodologies including entity-relationship modeling and normalization, and evaluates access control models and version control concepts applicable to parameter management.

Chapter \ref{chap:state-of-art} details the methodology and concept development, beginning with a thorough requirements analysis based on industry needs and academic best practices. This chapter explores the conceptual architecture design through use case modeling, compares different approaches for user management and parameter synchronization, and presents a comprehensive entity-relationship model for the system. Particular attention is given to validation mechanisms and integration approaches with existing enterprise systems.

Chapter 4 presents the implementation strategy and technical design of the\ac{VMAP} system. This chapter describes the practical realization of the database structure, detailing the core data entities, version control implementation, variant and segment management, access control mechanisms, query optimization approaches, change tracking implementation, and integration with external systems. The chapter provides concrete code examples showing the translation of conceptual designs into functional database components.

Chapter 5 focuses on system evaluation and validation, presenting a comprehensive assessment of the\ac{VMAP} system against the defined research objectives. This chapter describes the validation methodology used to evaluate the system, presents detailed functional testing results for user management, release management, and variant management capabilities, analyzes the system's performance characteristics including query optimization and storage requirements, and evaluates the versioning approach and integration capabilities.

The thesis concludes with Chapter 6, which summarizes the research findings, discusses the limitations of the current implementation, and presents recommendations for future development. This chapter reflects on the contributions of the research to both academic knowledge and industry practice, presenting both technical optimizations for future enhancement and broader implications for database research and automotive software development.

Throughout these chapters, the research methodology combines theoretical analysis with practical implementation, ensuring that the resulting system meets both academic standards and industry requirements. Special attention is given to database versioning approaches, user role management, and integration strategies with existing systems, addressing the unique challenges of automotive software configuration management \cite{staron2021automotive}.

\section{Project Plan}

The research project follows a structured approach spanning six months from November 2024 to April 2025, organized into three distinct phases: Exposé, Implementation, and Finalization. The comprehensive timeline ensures systematic progression through all research objectives while maintaining academic rigor and quality standards.

\subsection{Exposé Phase (November - January)}

The initial phase focuses on establishing strong theoretical foundations and gathering comprehensive requirements. Literature review constitutes a significant portion of this phase, extending over six weeks to ensure thorough coverage of current database versioning approaches, parameter management systems, and industry practices in automotive applications. This review encompasses analysis of existing version control systems, examination of industry standards for software configuration management, and evaluation of current parameter management solutions.

Requirements analysis follows the literature review, spanning three weeks to capture detailed system specifications. This phase involves extensive stakeholder consultation to document system requirements, analyze existing Excel-based workflows, define integration requirements with internal database systems, and establish user roles and access control specifications. The Exposé phase concludes with the submission of a comprehensive research proposal at the end of Week 3 in January.

\subsection{Implementation Phase (December - March)}

The implementation phase encompasses four major components, each allocated four weeks for development and refinement. Database design initiates this phase, focusing on developing the schema for parameter management, designing version control mechanisms, creating data models for user management, and planning integration interfaces with existing systems. 

System architecture development follows, concentrating on overall system design, version control workflows, user management frameworks, and validation mechanisms. This stage establishes the foundational structure for the entire system while ensuring alignment with identified requirements and industry standards.

Prototype development constitutes the third component, involving implementation of core database functionality, development of version control features, creation of user management interfaces, and construction of system integration components. This stage transforms theoretical designs into practical implementations while maintaining focus on system usability and performance.

The final component of this phase involves comprehensive testing and validation, including database performance testing, validation of version control mechanisms, testing of user management functions, and verification of system integration capabilities. This stage ensures all implemented features meet specified requirements and performance standards.

\subsection{Finalization Phase (April)}

The concluding phase focuses on documentation and thesis preparation over four weeks. The first two weeks are dedicated to comprehensive documentation, including compilation of implementation details and system architecture documentation. 

The subsequent two weeks concentrate on thesis writing, involving comprehensive documentation of research findings, inclusion of test results and analysis, preparation of conclusions and recommendations, and thorough content review and refinement. The phase concludes with thesis submission in Week 16 and final project presentation in Week 17.

\begin{sidewaysfigure}
        \begin{ganttchart}[vgrid,
            y unit title = 0.7cm,
            y unit chart = 0.5cm,
            bar/.append style={fill=red!70},
            bar incomplete/.append style={fill=black!30},
            bar top shift =0.2,
            bar height=.5,
            group/.append style={draw=black,fill=NavyBlue!70},
            group incomplete/.append style={draw=black, fill=black!50},
            group left shift=0,
            group right shift=0,
            group height=.7,
            group peaks height = 0,
            milestone height = 0.7,
            milestone/.append style={fill=orange},
            ]{1}{25}
            
            \gantttitle{2024-2025}{25} \\
            \gantttitle{Nov}{4}\gantttitle{Dec}{4}\gantttitle{Jan}{5}\gantttitle{Feb}{4}\gantttitle{Mar}{4}\gantttitle{Apr}{4} \\
            \gantttitlelist{45,...,52}{1}
            \gantttitlelist{1,...,17}{1} \\
            
            \ganttgroup{Exposé}{1}{7} \\
            \ganttbar{Literature Review}{1}{6} \\
            \ganttbar{Requirements Analysis}{3}{5} \\
            \ganttmilestone{Exposé Submission}{11} \\
            
            \ganttgroup{Implementation}{5}{21} \\
            \ganttbar{Database Design}{5}{8} \\
            \ganttlinkedbar{System Architecture}{9}{12} \\
            \ganttlinkedbar{Prototype Development}{13}{17} \\
            \ganttlinkedbar{Testing and Validation}{18}{21} \\
            
            \ganttgroup{Finalization}{21}{24} \\
            \ganttbar{Documentation}{21}{22} \\
            \ganttlinkedbar{Thesis Writing}{23}{24} \\
            \ganttmilestone{Final Submission}{24} \\
            \ganttmilestone{Presentation}{25} \\
            
            \\ \ganttnewline[thick, black]
        
        \end{ganttchart}
    \caption{Gantt Chart of the planned work schedule.}\label{fig:Gantt-chart-1}
\end{sidewaysfigure}