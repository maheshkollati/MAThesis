\chapter{Implementation}
\label{chap:implementation}

This chapter presents the technical implementation of the VMAP system conceptual architecture described in Chapter \ref{chap:methodology}. The discussion focuses on key architectural components and technical decisions that enable efficient parameter versioning in automotive software development, with concrete implementation details demonstrating how theoretical concepts translate to practical solutions.

\section{Database Structure Implementation}
\label{sec:database-structure-implementation}

Following the comparative analysis of database systems presented in Chapter \ref{chap:methodology}, PostgreSQL was selected as the implementation platform due to its superior support for complex data types, extensibility features, and advanced indexing capabilities \cite{obe2017postgresql}. The database implementation transforms abstract entities and relationships into concrete database structures, implementing the hierarchical organization of automotive electronic systems through physical tables and relationships.

\subsection{Core Data Entities}
\label{subsec:core-data-entities}

The hierarchical structure of automotive electronic systems is implemented through four primary entity types: ECUs, Modules, PIDs, and Parameters. The ECU and Module entities form the top levels of the hierarchy, with a many-to-many relationship reflecting the reality that modules can exist across multiple ECUs. This structure aligns with the domain model described by Staron \cite{staron2021automotive}, where logical groupings of software functions must be maintained across different hardware configurations.

\begin{lstlisting}[language=SQL, caption={ECU and Module Table Implementation}, label={lst:ecu-module-tables}]
CREATE TABLE ecus (
    ecu_id INTEGER PRIMARY KEY,
    name VARCHAR(100) NOT NULL,
    description TEXT,
    byte_order VARCHAR(50),
    created_at TIMESTAMP WITH TIME ZONE DEFAULT CURRENT_TIMESTAMP,
    external_id INTEGER UNIQUE -- PDD reference ID
);

CREATE TABLE modules (
    module_id INTEGER PRIMARY KEY,
    shortcut VARCHAR(50) NOT NULL,
    name VARCHAR(255) NOT NULL,
    kind VARCHAR(255),
    created_at TIMESTAMP WITH TIME ZONE DEFAULT CURRENT_TIMESTAMP,
    external_id INTEGER UNIQUE -- PDD reference ID
);

CREATE TABLE ecu_modules (
    ecu_id INTEGER REFERENCES ecus(ecu_id) ON DELETE CASCADE,
    module_id INTEGER REFERENCES modules(module_id) ON DELETE CASCADE,
    created_at TIMESTAMP WITH TIME ZONE DEFAULT CURRENT_TIMESTAMP,
    PRIMARY KEY (ecu_id, module_id)
);
\end{lstlisting}

The junction table \texttt{ecu\_modules} implements the many-to-many relationship using foreign key constraints as described by Elmasri and Navathe \cite{elmasri2015fundamentals}. Each entity includes an \texttt{external\_id} attribute to maintain mapping with the Parameter Definition Database (PDD), facilitating integration through persistent identifier correlation as recommended by Hohpe and Woolf \cite{hohpe2002enterprise}.

PIDs (Parameter IDs) and parameters constitute the lower levels of the hierarchy, with PIDs grouping related parameters for specific modules. The parameter table implements additional attributes for handling the complexity of automotive parameter specifications:

\begin{lstlisting}[language=SQL, caption={Parameter Table Implementation}, label={lst:parameter-table}]
CREATE TABLE parameters (
    parameter_id BIGINT PRIMARY KEY,
    pid_id BIGINT REFERENCES pids(pid_id) ON DELETE CASCADE,
    ecu_id INTEGER,
    phase_id INTEGER,
    name VARCHAR(255) NOT NULL,
    parameter_name VARCHAR(255),
    type_id INTEGER REFERENCES parameter_data_types(data_type_id),
    array_definition VARCHAR(50),
    position INTEGER,
    factor DECIMAL,
    unit VARCHAR(50),
    bias_offset DECIMAL,
    is_active BOOLEAN DEFAULT true,
    external_id INTEGER, -- PDD reference ID
    FOREIGN KEY (ecu_id, phase_id) REFERENCES ecu_phases(ecu_id, phase_id)
);
\end{lstlisting}

The parameter table incorporates a form of strategic denormalization by including direct references to \texttt{ecu\_id} and \texttt{phase\_id} alongside the PID foreign key. While this introduces some redundancy, this approach aims to improve query execution for parameter queries filtered by release phase, a critical operation in the system. Bhattacherjee et al. \cite{bhattacherjee2015principles} note that such denormalization can be justified when query performance benefits outweigh the overhead of maintaining consistency, especially for frequently accessed paths in the data model.

A particularly challenging aspect of the implementation was supporting multi-dimensional parameters, which are common in automotive applications for representing lookup tables and characteristic curves \cite{kiencke2000automotive}. This was addressed through a specialized table structure:

\begin{lstlisting}[language=SQL, caption={Parameter Dimension Implementation}, label={lst:parameter-dimension}]
CREATE TABLE parameter_dimensions (
    dimension_id BIGINT PRIMARY KEY,
    parameter_id BIGINT REFERENCES parameters(parameter_id) ON DELETE CASCADE,
    dimension_index INTEGER NOT NULL,
    default_value NUMERIC NOT NULL,
    external_id INTEGER, -- PDD reference ID
    UNIQUE (parameter_id, dimension_index)
);
\end{lstlisting}

This implementation follows a modified entity-attribute-value (EAV) pattern while addressing the potential challenges typically associated with such models \cite{nadkarni2016revisiting}. The introduction of a dimension index provides an ordered structure to parameter dimensions, enabling efficient representation of arrays and matrices while maintaining the relationship between parameters and their dimensional values.

\subsection{Version Control Implementation}
\label{subsec:version-control-implementation}

The version control implementation is a defining feature of the VMAP system, enabling parameter evolution management across different development phases. After evaluating multiple approaches described in Chapter \ref{chap:methodology}, the phase-based versioning model was implemented, creating explicit relationships between parameters and development phases rather than using a generic temporal approach.

\begin{lstlisting}[language=SQL, caption={Release and Phase Management Implementation}, label={lst:release-phase-tables}]
CREATE TABLE releases (
    release_id INTEGER PRIMARY KEY,
    name VARCHAR(50) NOT NULL UNIQUE, -- e.g., "24.1", "24.3"
    description TEXT,
    is_active BOOLEAN DEFAULT true,
    created_at TIMESTAMP WITH TIME ZONE DEFAULT CURRENT_TIMESTAMP,
    created_by BIGINT REFERENCES users(user_id)
);

CREATE TABLE release_phases (
    phase_id INTEGER PRIMARY KEY,
    release_id INTEGER REFERENCES releases(release_id) ON DELETE CASCADE,
    name VARCHAR(50) NOT NULL, -- e.g., "Initial", "PreTest1", "PreTest2", "Final"
    sequence_number INTEGER NOT NULL, -- Determines the order of phases
    is_active BOOLEAN DEFAULT true,
    created_at TIMESTAMP WITH TIME ZONE DEFAULT CURRENT_TIMESTAMP,
    created_by BIGINT REFERENCES users(user_id),
    UNIQUE (release_id, name),
    UNIQUE (release_id, sequence_number)
);

CREATE TABLE ecu_phases (
    ecu_id INTEGER REFERENCES ecus(ecu_id) ON DELETE CASCADE,
    phase_id INTEGER REFERENCES release_phases(phase_id) ON DELETE CASCADE,
    is_active BOOLEAN DEFAULT true,
    is_frozen BOOLEAN DEFAULT false,
    frozen_at TIMESTAMP WITH TIME ZONE,
    frozen_by BIGINT REFERENCES users(user_id),
    created_at TIMESTAMP WITH TIME ZONE DEFAULT CURRENT_TIMESTAMP,
    created_by BIGINT REFERENCES users(user_id),
    PRIMARY KEY (ecu_id, phase_id)
);
\end{lstlisting}

This implementation supports the bi-annual release cycle with four sequential phases per release. The \texttt{sequence\_number} field provides explicit ordering of phases within a release, while unique constraints ensure consistency of phase naming and sequencing. The ECU-phase mapping implements the association between ECUs and specific release phases, supporting independent progression of different ECUs through the development cycle as described by Broy \cite{broy2006challenges}.

The phase-based approach was selected over temporal versioning for several reasons supported by database design principles:

\begin{itemize}
    \item Direct phase associations align with the process-based nature of automotive development, where parameters evolve through explicitly defined development stages.
    \item The explicit phase relationships simplify the implementation of phase transitions and comparison operations, which are fundamental to the automotive development process as described by Pretschner et al. \cite{pretschner2007software}.
    \item The approach aligns with the mental model of automotive development engineers, who conceptualize parameter evolution in terms of distinct development phases rather than continuous time.
\end{itemize}

Phase management functionality was implemented through a stored procedure for phase transitions and parameter propagation:

\begin{lstlisting}[language=SQL, caption={Phase Transition Function}, label={lst:phase-transition}]
CREATE OR REPLACE FUNCTION copy_phase_data(
    source_ecu_id INTEGER,
    source_phase_id INTEGER,
    target_ecu_id INTEGER,
    target_phase_id INTEGER,
    user_id BIGINT
)
RETURNS TABLE (
    variants_copied INTEGER,
    segments_copied INTEGER
) AS $$
DECLARE
    variant_count INTEGER := 0;
    segment_count INTEGER := 0;
    transaction_id BIGINT;
BEGIN
    -- Get a transaction ID for change tracking
    SELECT nextval('change_history_transaction_id_seq') INTO transaction_id;
    
    -- Set the transaction ID for tracking in logging trigger
    PERFORM set_config('app.transaction_id', transaction_id::text, true);
    PERFORM set_config('app.user_id', user_id::text, true);
    
    -- Create a temporary table to map source variant IDs to target variant IDs
    CREATE TEMPORARY TABLE variant_mapping (
        source_variant_id BIGINT,
        target_variant_id BIGINT
    ) ON COMMIT DROP;
    
    -- Copy variants with modified attributes for target phase
    WITH inserted_variants AS (
        INSERT INTO variants (
            pid_id, ecu_id, phase_id, name, code_rule, 
            created_at, created_by, updated_by
        )
        SELECT 
            v.pid_id, target_ecu_id, target_phase_id, v.name, v.code_rule,
            CURRENT_TIMESTAMP, user_id, user_id
        FROM variants v
        WHERE v.ecu_id = source_ecu_id AND v.phase_id = source_phase_id
        RETURNING variant_id, pid_id, name, code_rule
    )
    INSERT INTO variant_mapping (source_variant_id, target_variant_id)
    SELECT v.variant_id, iv.variant_id
    FROM variants v
    JOIN inserted_variants iv ON v.pid_id = iv.pid_id 
                             AND v.name = iv.name 
                             AND v.code_rule = iv.code_rule
    WHERE v.ecu_id = source_ecu_id AND v.phase_id = source_phase_id;
    
    -- Get count of copied variants
    SELECT COUNT(*) INTO variant_count FROM variant_mapping;
    
    -- Copy segments with parameter mapping
    -- [Additional implementation details omitted for brevity]
    
    RETURN QUERY SELECT variant_count, segment_count;
END;
$$ LANGUAGE plpgsql;
\end{lstlisting}

This implementation provides an atomic transaction for phase transition, copying parameter variants and segments from one phase to another while maintaining proper references and tracking the operation in the change history. The function returns counts of affected entities, providing immediate feedback on the operation's scope. The use of temporary tables for mapping entities between phases implements the identity map pattern described by Fowler \cite{fowler2003patterns}, ensuring proper relationship preservation during complex operations.

\subsection{Variant and Segment Management}
\label{subsec:variant-segment-management}

The variant and segment management implementation realizes the core parameter customization capabilities of the VMAP system:

\begin{lstlisting}[language=SQL, caption={Variant and Segment Implementation}, label={lst:variant-segment-tables}]
CREATE TABLE variants (
    variant_id BIGINT PRIMARY KEY,
    pid_id BIGINT REFERENCES pids(pid_id) ON DELETE CASCADE,
    ecu_id INTEGER,
    phase_id INTEGER,
    name VARCHAR(100) NOT NULL,
    code_rule TEXT,
    created_at TIMESTAMP WITH TIME ZONE DEFAULT CURRENT_TIMESTAMP,
    created_by BIGINT REFERENCES users(user_id),
    updated_by BIGINT REFERENCES users(user_id),
    FOREIGN KEY (ecu_id, phase_id) REFERENCES ecu_phases(ecu_id, phase_id),
    UNIQUE (phase_id, pid_id, name),
    UNIQUE (phase_id, pid_id, code_rule)
);

CREATE TABLE segments (
    segment_id BIGINT PRIMARY KEY,
    variant_id BIGINT REFERENCES variants(variant_id) ON DELETE CASCADE,
    parameter_id BIGINT REFERENCES parameters(parameter_id) ON DELETE CASCADE,
    dimension_index INTEGER NOT NULL,
    decimal NUMERIC NOT NULL,
    created_at TIMESTAMP WITH TIME ZONE DEFAULT CURRENT_TIMESTAMP,
    created_by BIGINT REFERENCES users(user_id),
    updated_by BIGINT REFERENCES users(user_id)
);
\end{lstlisting}

Each variant is associated with a specific PID, ECU, and phase, creating a three-way relationship that places the variant in both functional and temporal contexts. The uniqueness constraints prevent duplicate variant names or code rules within the same PID and phase, enforcing a critical business rule identified during requirements analysis. The \texttt{code\_rule} field stores boolean expressions determining when a variant applies based on vehicle configuration codes, implementing a domain-specific rule language for variant applicability.

Segments form the foundation of parameter customization, linking parameters to variants and storing modified values. The canonical decimal representation for all parameter values, regardless of native data type, implements the canonical model pattern described by Hohpe and Woolf \cite{hohpe2002enterprise}, simplifying data manipulation while ensuring consistent value handling.

For parameter value resolution, a specialized database function was implemented:

\begin{lstlisting}[language=SQL, caption={Parameter Resolution Function}, label={lst:parameter-resolution}]
CREATE OR REPLACE FUNCTION resolve_parameter_value(
    p_parameter_id BIGINT,
    p_dimension_index INTEGER,
    p_variant_ids BIGINT[]
)
RETURNS NUMERIC AS $$
DECLARE
    v_value NUMERIC;
    v_default_value NUMERIC;
BEGIN
    -- First try to find a segment with the given parameter and variant
    SELECT s.decimal INTO v_value
    FROM segments s
    WHERE s.parameter_id = p_parameter_id
      AND s.dimension_index = p_dimension_index
      AND s.variant_id = ANY(p_variant_ids)
    ORDER BY array_position(p_variant_ids, s.variant_id)
    LIMIT 1;
    
    -- If no segment found, get the default value
    IF v_value IS NULL THEN
        SELECT pd.default_value INTO v_default_value
        FROM parameter_dimensions pd
        WHERE pd.parameter_id = p_parameter_id
          AND pd.dimension_index = p_dimension_index;
          
        -- If no dimension record, try parameter default
        IF v_default_value IS NULL THEN
            SELECT p.default_value INTO v_default_value
            FROM parameters p
            WHERE p.parameter_id = p_parameter_id;
        END IF;
        
        v_value := COALESCE(v_default_value, 0);
    END IF;
    
    RETURN v_value;
END;
$$ LANGUAGE plpgsql;
\end{lstlisting}

This function implements the value resolution logic required for parameter file generation, handling the complex logic of finding applicable parameter values based on variant precedence. The array position-based ordering ensures that variants are applied in the correct priority sequence, a critical requirement for automotive parameter management as described by Staron \cite{staron2021automotive}.

To support documentation and compliance requirements, a snapshot mechanism was implemented to capture complete parameter configurations at specific points in time:

\begin{lstlisting}[language=SQL, caption={Documentation Snapshot Implementation}, label={lst:documentation-snapshot}]
CREATE TABLE documentation_snapshots (
    snapshot_id INTEGER PRIMARY KEY,
    name VARCHAR(255) NOT NULL,
    description TEXT,
    ecu_id INTEGER,
    phase_id INTEGER,
    variant_count INTEGER DEFAULT 0,
    segment_count INTEGER DEFAULT 0,
    created_at TIMESTAMP WITH TIME ZONE DEFAULT CURRENT_TIMESTAMP,
    created_by BIGINT REFERENCES users(user_id),
    FOREIGN KEY (ecu_id, phase_id) REFERENCES ecu_phases(ecu_id, phase_id)
);

CREATE TABLE snapshot_variants (
    snapshot_variant_id INTEGER PRIMARY KEY,
    snapshot_id INTEGER REFERENCES documentation_snapshots(snapshot_id) ON DELETE CASCADE,
    original_variant_id BIGINT, -- Reference to the original variant
    pid_id BIGINT REFERENCES pids(pid_id) ON DELETE CASCADE,
    name VARCHAR(100) NOT NULL,
    code_rule TEXT,
    created_at TIMESTAMP WITH TIME ZONE,
    created_by BIGINT REFERENCES users(user_id)
);

CREATE TABLE snapshot_segments (
    snapshot_segment_id INTEGER PRIMARY KEY,
    snapshot_id INTEGER REFERENCES documentation_snapshots(snapshot_id) ON DELETE CASCADE,
    snapshot_variant_id INTEGER REFERENCES snapshot_variants(snapshot_variant_id) ON DELETE CASCADE,
    original_segment_id BIGINT, -- Reference to the original segment
    parameter_id BIGINT REFERENCES parameters(parameter_id) ON DELETE CASCADE,
    dimension_index INTEGER NOT NULL,
    decimal NUMERIC NOT NULL,
    created_at TIMESTAMP WITH TIME ZONE,
    created_by BIGINT REFERENCES users(user_id)
);
\end{lstlisting}

This implementation follows the snapshot pattern described by Fowler \cite{fowler2003patterns}, creating a complete copy of variant and segment data at specific points in time. Rather than using a temporal database approach with validity periods, the system explicitly materializes historical states, ensuring they remain accessible regardless of subsequent modifications to live data. References to original entities enable traceability between snapshot and live data, implementing the origin tracking pattern described by Tichy \cite{tichy1985rcs}.

The snapshot creation function automates the process of capturing parameter configurations:

\begin{lstlisting}[language=SQL, caption={Documentation Snapshot Function}, label={lst:documentation-snapshot-function}]
CREATE OR REPLACE FUNCTION create_documentation_snapshot(
    p_name VARCHAR(255),
    p_description TEXT,
    p_ecu_id INTEGER,
    p_phase_id INTEGER,
    p_user_id BIGINT
)
RETURNS INTEGER AS $$
DECLARE
    v_snapshot_id INTEGER;
BEGIN
    -- Create the snapshot record
    INSERT INTO documentation_snapshots (
        name, description, ecu_id, phase_id,
        variant_count, segment_count, created_at, created_by
    ) VALUES (
        p_name, p_description, p_ecu_id, p_phase_id,
        0, 0, CURRENT_TIMESTAMP, p_user_id
    ) RETURNING snapshot_id INTO v_snapshot_id;

    -- Copy all variants for the ECU and phase
    INSERT INTO snapshot_variants (
        snapshot_id, original_variant_id, pid_id, name,
        code_rule, created_at, created_by
    )
    SELECT 
        v_snapshot_id, v.variant_id, v.pid_id, v.name,
        v.code_rule, v.created_at, v.created_by
    FROM variants v
    WHERE v.ecu_id = p_ecu_id AND v.phase_id = p_phase_id;
    
    -- Copy all segments for the variants
    -- [Implementation details omitted for brevity]
    
    -- Update the counts in the snapshot record
    -- [Implementation details omitted for brevity]
    
    RETURN v_snapshot_id;
END;
$$ LANGUAGE plpgsql;
\end{lstlisting}

In the automotive industry, these snapshots serve multiple purposes: providing immutable records of parameter configurations at significant development milestones, supporting quality assurance processes and regulatory compliance requirements, and facilitating comparative analysis between development phases.

\section{Access Control Implementation}
\label{sec:access-control-implementation}

The access control implementation realizes the hybrid role-permission model described in Chapter \ref{chap:methodology}, providing a flexible foundation for managing user permissions across the system.

\subsection{Role-Based Permission Model}
\label{subsec:role-based-permission-model}

The core RBAC implementation follows the structure defined by Sandhu et al. \cite{sandhu1998role}, with tables for users, roles, permissions, and their relationships:

\begin{lstlisting}[language=SQL, caption={Core RBAC Implementation}, label={lst:rbac-implementation}]
CREATE TABLE roles (
    role_id INTEGER PRIMARY KEY,
    name VARCHAR(255) UNIQUE NOT NULL,
    description TEXT
);

CREATE TABLE permissions (
    permission_id INTEGER PRIMARY KEY,
    name VARCHAR(255) UNIQUE NOT NULL,
    description TEXT
);

CREATE TABLE role_permissions (
    role_id INTEGER REFERENCES roles(role_id) ON DELETE CASCADE,
    permission_id INTEGER REFERENCES permissions(permission_id) ON DELETE CASCADE,
    granted_at TIMESTAMP WITH TIME ZONE DEFAULT CURRENT_TIMESTAMP,
    granted_by BIGINT REFERENCES users(user_id),
    PRIMARY KEY (role_id, permission_id)
);

CREATE TABLE user_roles (
    user_id BIGINT REFERENCES users(user_id) ON DELETE CASCADE,
    role_id INTEGER REFERENCES roles(role_id) ON DELETE CASCADE,
    granted_at TIMESTAMP WITH TIME ZONE DEFAULT CURRENT_TIMESTAMP,
    granted_by BIGINT REFERENCES users(user_id),
    PRIMARY KEY (user_id, role_id)
);
\end{lstlisting}

The model is extended with direct user permissions to implement the hybrid approach:

\begin{lstlisting}[language=SQL, caption={User-Permission Implementation}, label={lst:user-permission}]
CREATE TABLE user_permissions (
    user_permission_id INTEGER PRIMARY KEY,
    user_id BIGINT REFERENCES users(user_id) ON DELETE CASCADE,
    permission_id INTEGER REFERENCES permissions(permission_id) ON DELETE CASCADE,
    granted_by BIGINT REFERENCES users(user_id),
    granted_at TIMESTAMP WITH TIME ZONE DEFAULT CURRENT_TIMESTAMP,
    UNIQUE (user_id, permission_id)
);
\end{lstlisting}

This hybrid model allows permissions to be granted either through role assignments or directly to users, providing the flexibility to handle exceptional cases without creating specialized roles, as recommended by Ferraiolo et al. \cite{ferraiolo2011policy}.

The permission checking function implements the core security verification logic:

\begin{lstlisting}[language=SQL, caption={Permission Check Implementation}, label={lst:permission-check}]
CREATE OR REPLACE FUNCTION has_permission(
    p_user_id BIGINT,
    p_permission_name VARCHAR
)
RETURNS BOOLEAN AS $$
BEGIN
    RETURN EXISTS (
        -- Check role-based permissions
        SELECT 1
        FROM user_roles ur
        JOIN role_permissions rp ON ur.role_id = rp.role_id
        JOIN permissions p ON rp.permission_id = p.permission_id
        WHERE ur.user_id = p_user_id AND p.name = p_permission_name
        
        UNION
        
        -- Check direct user permissions
        SELECT 1
        FROM user_permissions up
        JOIN permissions p ON up.permission_id = p.permission_id
        WHERE up.user_id = p_user_id AND p.name = p_permission_name
    );
END;
$$ LANGUAGE plpgsql;
\end{lstlisting}

The UNION-based approach combines role-based and direct permission checks in a single database query, implementing an efficient verification mechanism. This implementation supports the separation of permission checking from application logic, allowing security policies to be enforced consistently across different access paths.

\subsection{Module-Based Access Control}
\label{subsec:module-based-access-control}

To complement the role-based permission model, a module-based access control system was implemented to restrict write access based on module assignments:

\begin{lstlisting}[language=SQL, caption={Module Access Control Implementation}, label={lst:module-access}]
CREATE TABLE user_access (
    user_id BIGINT REFERENCES users(user_id) ON DELETE CASCADE,
    ecu_id INTEGER REFERENCES ecus(ecu_id) ON DELETE CASCADE,
    module_id INTEGER REFERENCES modules(module_id) ON DELETE CASCADE,
    write_access BOOLEAN DEFAULT true,
    created_at TIMESTAMP WITH TIME ZONE DEFAULT CURRENT_TIMESTAMP,
    created_by BIGINT REFERENCES users(user_id),
    PRIMARY KEY (user_id, ecu_id, module_id),
    CONSTRAINT user_access_ecu_module_fk FOREIGN KEY (ecu_id, module_id) 
    REFERENCES ecu_modules(ecu_id, module_id)
);
\end{lstlisting}

This implementation establishes a three-way relationship between users, ECUs, and modules, with a boolean flag distinguishing between read and write access. The constraint ensures access is granted only for valid ECU-module combinations, enforcing structural integrity. Database functions verify both role-based permissions and module-based access rights during critical operations:

\begin{lstlisting}[language=SQL, caption={Module Access Check Implementation}, label={lst:module-access-check}]
CREATE OR REPLACE FUNCTION has_module_write_access(
    p_user_id BIGINT,
    p_ecu_id INTEGER,
    p_module_id INTEGER
)
RETURNS BOOLEAN AS $$
BEGIN
    -- First check if user has administrator role (bypasses module restrictions)
    IF EXISTS (
        SELECT 1
        FROM user_roles ur
        JOIN roles r ON ur.role_id = r.role_id
        WHERE ur.user_id = p_user_id AND r.name = 'administrator'
    ) THEN
        RETURN TRUE;
    END IF;
    
    -- Check module-specific write access
    RETURN EXISTS (
        SELECT 1
        FROM user_access ua
        WHERE ua.user_id = p_user_id
          AND ua.ecu_id = p_ecu_id
          AND ua.module_id = p_module_id
          AND ua.write_access = TRUE
    );
END;
$$ LANGUAGE plpgsql;
\end{lstlisting}

This implementation combines role-based and attribute-based access control approaches, creating what Ferraiolo et al. describe as policy-enhanced RBAC \cite{ferraiolo2011policy}. The function first checks for administrator privileges, which bypass module-specific restrictions, then verifies specific module access rights if necessary.

\section{Query Optimization Implementation}
\label{sec:query-optimization-implementation}

The complex relationships and large data volumes in automotive parameter management necessitated a comprehensive indexing strategy to support common query patterns.

\subsection{Indexing Implementation}
\label{subsec:indexing-implementation}

The indexing strategy focuses on accelerating hierarchical navigation and parameter retrieval operations:

\begin{lstlisting}[language=SQL, caption={Core Index Implementation}, label={lst:core-indexes}]
-- Hierarchical path indexes
CREATE INDEX idx_pids_ecu_module ON pids(ecu_id, module_id);
CREATE INDEX idx_parameters_pid_phase ON parameters(pid_id, phase_id);
CREATE INDEX idx_variants_pid_phase ON variants(pid_id, phase_id);
CREATE INDEX idx_segments_variant ON segments(variant_id);
CREATE INDEX idx_segments_parameter ON segments(parameter_id);

-- Phase-specific indexes
CREATE INDEX idx_parameters_phase ON parameters(phase_id);
CREATE INDEX idx_variants_phase ON variants(phase_id);

-- Text search optimization
CREATE EXTENSION IF NOT EXISTS pg_trgm;
CREATE INDEX idx_parameters_name_trgm 
    ON parameters USING gin (name gin_trgm_ops);
CREATE INDEX idx_variants_name_trgm 
    ON variants USING gin (name gin_trgm_ops);
\end{lstlisting}

These indexes match the natural navigation paths in the data model, implementing the access path optimization pattern described by Molinaro \cite{molinaro2005sql}. The hierarchical path indexes support efficient traversal from ECUs through modules and PIDs to parameters, while the phase-specific indexes accelerate queries filtered by release phase, a common operation in the system.

The trigram-based text search indexes utilize PostgreSQL's specialized text search capabilities as described by Obe and Hsu \cite{obe2017postgresql}. These indexes enable efficient pattern matching and similarity-based searches, accommodating the fuzzy search requirements identified during user interviews. According to Karwin \cite{karwin2010sql}, these specialized indexes can significantly improve the performance of text-based queries, which are common in parameter management systems where engineers need to locate parameters by name or partial name.

\subsection{Function-Based Optimization}
\label{subsec:function-based-optimization}

To optimize complex parameter retrieval operations, several specialized database functions were implemented:

\begin{lstlisting}[language=SQL, caption={Parameter Search Function}, label={lst:parameter-search}]
CREATE OR REPLACE FUNCTION search_parameters(
    search_term TEXT,
    max_results INT DEFAULT 20,
    ecu_id_filter INT DEFAULT NULL,
    phase_id_filter INT DEFAULT NULL
)
RETURNS TABLE (
    parameter_id BIGINT,
    name VARCHAR,
    parameter_name VARCHAR, 
    pid_id BIGINT,
    pid_name VARCHAR,
    module_id INTEGER,
    module_name VARCHAR,
    ecu_id INTEGER,
    ecu_name VARCHAR,
    phase_id INTEGER,
    relevance REAL
) AS $$
BEGIN
  RETURN QUERY
  SELECT 
    p.parameter_id,
    p.name,
    p.parameter_name,
    pid.pid_id,
    pid.name AS pid_name,
    m.module_id,
    m.name AS module_name,
    e.ecu_id,
    e.name AS ecu_name,
    p.phase_id,
    -- Calculate relevance score based on different matching criteria
    (CASE 
        -- Exact matches get highest score
        WHEN p.name ILIKE search_term OR p.parameter_name ILIKE search_term THEN 1.0
        -- Starts with gets high score
        WHEN p.name ILIKE (search_term || '%') OR p.parameter_name ILIKE (search_term || '%') THEN 0.8
        -- Contains gets medium score
        ELSE GREATEST(
            similarity(p.name, search_term),
            similarity(p.parameter_name, search_term)
        )
    END) AS relevance
  FROM 
    parameters p
  JOIN 
    pids pid ON p.pid_id = pid.pid_id
  JOIN 
    modules m ON pid.module_id = m.module_id
  JOIN 
    ecus e ON pid.ecu_id = e.ecu_id
  WHERE 
    -- Apply filters if provided
    (ecu_id_filter IS NULL OR e.ecu_id = ecu_id_filter) AND
    (phase_id_filter IS NULL OR p.phase_id = phase_id_filter) AND
    -- Match parameter name or display name
    (p.name ILIKE '%' || search_term || '%' OR 
     p.parameter_name ILIKE '%' || search_term || '%') AND
    -- Only active parameters
    p.is_active = true
  ORDER BY 
    relevance DESC, 
    p.name ASC
  LIMIT max_results;
END;
$$ LANGUAGE plpgsql;
\end{lstlisting}

This function implements a sophisticated parameter search algorithm that combines exact matches, prefix matches, and similarity-based matching to provide relevant search results. The relevance scoring approach ensures that the most relevant parameters appear first in the results, following the information retrieval principles described by Obe and Hsu \cite{obe2017postgresql}.

For phase comparison operations, a specialized function implements the parameter difference detection logic:

\begin{lstlisting}[language=SQL, caption={Phase Comparison Function}, label={lst:phase-comparison}]
CREATE OR REPLACE FUNCTION get_pids_with_changes_between_phases(
    p_ecu_id INTEGER,
    p_module_id INTEGER,
    p_source_phase_id INTEGER,
    p_target_phase_id INTEGER
)
RETURNS TABLE (
    pid_id BIGINT,
    name VARCHAR,
    has_param_changes BOOLEAN,
    has_variant_changes BOOLEAN,
    has_segment_changes BOOLEAN
) AS $
BEGIN
    RETURN QUERY
    WITH common_pids AS (
        -- Get PIDs available in at least one of the phases
        SELECT DISTINCT p.pid_id, p.name
        FROM pids p
        LEFT JOIN pid_phases pp1 ON p.pid_id = pp1.pid_id AND pp1.phase_id = p_source_phase_id
        LEFT JOIN pid_phases pp2 ON p.pid_id = pp2.pid_id AND pp2.phase_id = p_target_phase_id
        WHERE p.ecu_id = p_ecu_id AND p.module_id = p_module_id
          AND (pp1.phase_id IS NOT NULL OR pp2.phase_id IS NOT NULL)
    ),
    -- Parameter changes detection
    param_changes AS (
        SELECT cp.pid_id
        FROM common_pids cp
        WHERE EXISTS (
            -- Parameters that exist in target but not in source
            SELECT 1 FROM parameters pt
            WHERE pt.pid_id = cp.pid_id
              AND pt.phase_id = p_target_phase_id
              AND NOT EXISTS (
                  SELECT 1 FROM parameters ps
                  WHERE ps.pid_id = cp.pid_id
                    AND ps.phase_id = p_source_phase_id
                    AND ps.external_id = pt.external_id
              )
        ) OR EXISTS (
            -- Parameters that exist in source but not in target
            SELECT 1 FROM parameters ps
            WHERE ps.pid_id = cp.pid_id
              AND ps.phase_id = p_source_phase_id
              AND NOT EXISTS (
                  SELECT 1 FROM parameters pt
                  WHERE pt.pid_id = cp.pid_id
                    AND pt.phase_id = p_target_phase_id
                    AND pt.external_id = ps.external_id
              )
        ) OR EXISTS (
            -- Parameters that exist in both but have differences
            SELECT 1 FROM parameters ps
            JOIN parameters pt ON ps.external_id = pt.external_id
            WHERE ps.pid_id = cp.pid_id
              AND pt.pid_id = cp.pid_id
              AND ps.phase_id = p_source_phase_id
              AND pt.phase_id = p_target_phase_id
              AND (
                  ps.name != pt.name OR
                  ps.parameter_name != pt.parameter_name OR
                  ps.type_id != pt.type_id OR
                  ps.factor != pt.factor OR
                  ps.bias_offset != pt.bias_offset OR
                  ps.unit != pt.unit
              )
        )
    ),
    -- Variant changes detection
    variant_changes AS (
        SELECT cp.pid_id
        FROM common_pids cp
        WHERE EXISTS (
            -- Different variant counts between phases
            SELECT 1
            FROM (
                SELECT pid_id, count(*) as cnt
                FROM variants
                WHERE pid_id = cp.pid_id AND phase_id = p_source_phase_id
                GROUP BY pid_id
                
                UNION ALL
                
                SELECT pid_id, -count(*) as cnt
                FROM variants
                WHERE pid_id = cp.pid_id AND phase_id = p_target_phase_id
                GROUP BY pid_id
            ) v
            GROUP BY v.pid_id
            HAVING sum(v.cnt) != 0
        ) OR EXISTS (
            -- Variants with same name but different code rules
            SELECT 1
            FROM variants vs
            JOIN variants vt ON vs.name = vt.name
            WHERE vs.pid_id = cp.pid_id
              AND vt.pid_id = cp.pid_id
              AND vs.phase_id = p_source_phase_id
              AND vt.phase_id = p_target_phase_id
              AND vs.code_rule != vt.code_rule
        )
    ),
    -- Segment changes detection
    segment_changes AS (
        SELECT cp.pid_id
        FROM common_pids cp
        JOIN parameters ps ON ps.pid_id = cp.pid_id AND ps.phase_id = p_source_phase_id
        JOIN parameters pt ON pt.pid_id = cp.pid_id AND pt.phase_id = p_target_phase_id
                         AND pt.external_id = ps.external_id
        JOIN variants vs ON vs.pid_id = cp.pid_id AND vs.phase_id = p_source_phase_id
        JOIN variants vt ON vt.pid_id = cp.pid_id AND vt.phase_id = p_target_phase_id
                       AND vt.name = vs.name
        JOIN segments ss ON ss.parameter_id = ps.parameter_id AND ss.variant_id = vs.variant_id
        JOIN segments st ON st.parameter_id = pt.parameter_id AND st.variant_id = vt.variant_id
                       AND st.dimension_index = ss.dimension_index
        WHERE ss.decimal != st.decimal
        
        UNION
        
        -- Segments in source not in target
        SELECT cp.pid_id
        FROM common_pids cp
        JOIN parameters ps ON ps.pid_id = cp.pid_id AND ps.phase_id = p_source_phase_id
        JOIN parameters pt ON pt.pid_id = cp.pid_id AND pt.phase_id = p_target_phase_id
                         AND pt.external_id = ps.external_id
        JOIN variants vs ON vs.pid_id = cp.pid_id AND vs.phase_id = p_source_phase_id
        JOIN variants vt ON vt.pid_id = cp.pid_id AND vt.phase_id = p_target_phase_id
                       AND vt.name = vs.name
        JOIN segments ss ON ss.parameter_id = ps.parameter_id AND ss.variant_id = vs.variant_id
        WHERE NOT EXISTS (
            SELECT 1 FROM segments st
            WHERE st.parameter_id = pt.parameter_id
              AND st.variant_id = vt.variant_id
              AND st.dimension_index = ss.dimension_index
        )
        
        UNION
        
        -- Segments in target not in source
        SELECT cp.pid_id
        FROM common_pids cp
        JOIN parameters ps ON ps.pid_id = cp.pid_id AND ps.phase_id = p_source_phase_id
        JOIN parameters pt ON pt.pid_id = cp.pid_id AND pt.phase_id = p_target_phase_id
                         AND pt.external_id = ps.external_id
        JOIN variants vs ON vs.pid_id = cp.pid_id AND vs.phase_id = p_source_phase_id
        JOIN variants vt ON vt.pid_id = cp.pid_id AND vt.phase_id = p_target_phase_id
                       AND vt.name = vs.name
        JOIN segments st ON st.parameter_id = pt.parameter_id AND st.variant_id = vt.variant_id
        WHERE NOT EXISTS (
            SELECT 1 FROM segments ss
            WHERE ss.parameter_id = ps.parameter_id
              AND ss.variant_id = vs.variant_id
              AND ss.dimension_index = st.dimension_index
        )
    )
    -- Return PIDs with change indicators
    SELECT 
        cp.pid_id,
        cp.name,
        (cp.pid_id IN (SELECT pid_id FROM param_changes)) AS has_param_changes,
        (cp.pid_id IN (SELECT pid_id FROM variant_changes)) AS has_variant_changes,
        (cp.pid_id IN (SELECT pid_id FROM segment_changes)) AS has_segment_changes
    FROM 
        common_pids cp
    WHERE
        -- Only return PIDs that have at least one type of change
        cp.pid_id IN (SELECT pid_id FROM param_changes)
        OR cp.pid_id IN (SELECT pid_id FROM variant_changes)
        OR cp.pid_id IN (SELECT pid_id FROM segment_changes)
    ORDER BY 
        cp.name;
END;
$ LANGUAGE plpgsql;
\end{lstlisting}

This function implements an efficient algorithm for identifying differences between parameter configurations in different release phases. Unlike a naive approach that would require retrieving and comparing all parameters, variants, and segments, this function performs the comparison directly in the database using specialized subqueries for each type of change. This implementation follows the set-based processing pattern recommended by Date \cite{date2011sql}, leveraging the database engine's capabilities for efficient data comparison.

\section{Change Tracking Implementation}
\label{sec:change-tracking-implementation}

The change tracking implementation provides comprehensive audit capabilities for all parameter data modifications, addressing both regulatory compliance requirements and supporting diagnostic analysis of parameter evolution.

\subsection{Automatic Change Logging}
\label{subsec:automatic-change-logging}

A database trigger mechanism automatically records all modifications to critical entities:

\begin{lstlisting}[language=SQL, caption={Change Tracking Trigger}, label={lst:change-tracking-trigger}]
CREATE TABLE change_history (
    change_id BIGINT PRIMARY KEY,
    user_id BIGINT REFERENCES users(user_id),
    ecu_id INTEGER,
    phase_id INTEGER,
    entity_type VARCHAR(50) NOT NULL,
    entity_id BIGINT NOT NULL,
    change_type VARCHAR(50),
    old_values JSONB,
    new_values JSONB,
    changed_at TIMESTAMP WITH TIME ZONE DEFAULT CURRENT_TIMESTAMP,
    transaction_id BIGINT NOT NULL
);

CREATE OR REPLACE FUNCTION log_change()
RETURNS TRIGGER AS $
DECLARE
    transaction_id BIGINT;
    change_type VARCHAR(50);
    old_values JSONB;
    new_values JSONB;
    entity_id BIGINT;
    phase_id INTEGER;
    ecu_id INTEGER;
BEGIN
    -- Get the user ID from the application context
    user_id := NULLIF(current_setting('app.user_id', TRUE), '')::BIGINT;
    
    -- Get the current transaction ID or create a new one
    SELECT COALESCE(NULLIF(current_setting('app.transaction_id', TRUE), '')::BIGINT, 
                   nextval('change_history_transaction_id_seq')) INTO transaction_id;
    
    -- Set the transaction ID for other triggers in the same transaction
    PERFORM set_config('app.transaction_id', transaction_id::text, TRUE);
    
    -- Determine entity_id and context information based on table and operation
    CASE TG_TABLE_NAME 
        WHEN 'variants' THEN 
            entity_id := CASE WHEN TG_OP = 'DELETE' THEN OLD.variant_id ELSE NEW.variant_id END;
            ecu_id := CASE WHEN TG_OP = 'DELETE' THEN OLD.ecu_id ELSE NEW.ecu_id END;
            phase_id := CASE WHEN TG_OP = 'DELETE' THEN OLD.phase_id ELSE NEW.phase_id END;
            
        WHEN 'segments' THEN 
            entity_id := CASE WHEN TG_OP = 'DELETE' THEN OLD.segment_id ELSE NEW.segment_id END;
            
            -- Get phase_id from associated variant
            IF TG_OP = 'DELETE' THEN
                SELECT v.phase_id, v.ecu_id INTO phase_id, ecu_id 
                FROM variants v WHERE v.variant_id = OLD.variant_id;
            ELSE
                SELECT v.phase_id, v.ecu_id INTO phase_id, ecu_id 
                FROM variants v WHERE v.variant_id = NEW.variant_id;
            END IF;
            
        -- Additional entity types handled here
        -- [Implementation details omitted for brevity]
    END CASE;
    
    -- Determine change type and capture entity state
    IF TG_OP = 'INSERT' THEN
        change_type := 'CREATE';
        old_values := NULL;
        new_values := to_jsonb(NEW);
    ELSIF TG_OP = 'UPDATE' THEN
        change_type := 'UPDATE';
        old_values := to_jsonb(OLD);
        new_values := to_jsonb(NEW);
    ELSIF TG_OP = 'DELETE' THEN
        change_type := 'DELETE';
        old_values := to_jsonb(OLD);
        new_values := NULL;
    END IF;
    
    -- Remove large or sensitive fields from the JSON
    IF old_values IS NOT NULL THEN
        old_values := old_values - 'created_at' - 'updated_at';
    END IF;
    
    IF new_values IS NOT NULL THEN
        new_values := new_values - 'created_at' - 'updated_at';
    END IF;
    
    -- Insert into change_history
    INSERT INTO change_history (
        user_id, ecu_id, phase_id, entity_type, entity_id,
        change_type, old_values, new_values, transaction_id, changed_at
    ) VALUES (
        user_id, ecu_id, phase_id, TG_TABLE_NAME, entity_id,
        change_type, old_values, new_values, transaction_id, CURRENT_TIMESTAMP
    );
    
    RETURN NULL;
END;
$ LANGUAGE plpgsql;

-- Apply the trigger to critical entities
CREATE TRIGGER variants_change_trigger
AFTER INSERT OR UPDATE OR DELETE ON variants
FOR EACH ROW EXECUTE FUNCTION log_change();

CREATE TRIGGER segments_change_trigger
AFTER INSERT OR UPDATE OR DELETE ON segments
FOR EACH ROW EXECUTE FUNCTION log_change();

-- Additional triggers for other entities
-- [Implementation details omitted for brevity]
\end{lstlisting}

This implementation captures complete entity states rather than just modified fields, storing them as JSONB documents. JSON path operators enable efficient extraction and querying of specific changes without complex joins, leveraging PostgreSQL's advanced document storage capabilities as described by Obe and Hsu \cite{obe2017postgresql}. The automatic trigger system ensures consistent logging regardless of how changes are made, preventing circumvention of audit controls.

The change tracking system implements several key patterns described by Fowler \cite{fowler2003patterns}: the unit of work pattern through transaction grouping, the state snapshot pattern through complete entity state capture, and the audit log pattern through comprehensive change recording. These patterns collectively support the sophisticated audit requirements common in regulated industries like automotive development.

\subsection{Change Analysis Functions}
\label{subsec:change-analysis-functions}

Specialized functions enable analysis of parameter evolution over time:

\begin{lstlisting}[language=SQL, caption={Parameter History Function}, label={lst:parameter-history}]
CREATE OR REPLACE FUNCTION get_parameter_history(
    p_parameter_id BIGINT,
    p_start_date TIMESTAMP WITH TIME ZONE DEFAULT NULL,
    p_end_date TIMESTAMP WITH TIME ZONE DEFAULT NULL
)
RETURNS TABLE (
    change_id BIGINT,
    changed_at TIMESTAMP WITH TIME ZONE,
    user_name VARCHAR,
    variant_name VARCHAR,
    old_value NUMERIC,
    new_value NUMERIC,
    change_type VARCHAR,
    transaction_id BIGINT,
    phase_name VARCHAR
) AS $
BEGIN
    RETURN QUERY
    SELECT 
        ch.change_id,
        ch.changed_at,
        u.first_name || ' ' || u.last_name AS user_name,
        v.name AS variant_name,
        (ch.old_values->>'decimal')::NUMERIC AS old_value,
        (ch.new_values->>'decimal')::NUMERIC AS new_value,
        ch.change_type,
        ch.transaction_id,
        rp.name AS phase_name
    FROM 
        change_history ch
        JOIN users u ON ch.user_id = u.user_id
        JOIN segments s ON ch.entity_id = s.segment_id 
                      AND ch.entity_type = 'segments'
        JOIN variants v ON s.variant_id = v.variant_id
        JOIN release_phases rp ON ch.phase_id = rp.phase_id
    WHERE 
        s.parameter_id = p_parameter_id
        AND (p_start_date IS NULL OR ch.changed_at >= p_start_date)
        AND (p_end_date IS NULL OR ch.changed_at <= p_end_date)
    ORDER BY 
        ch.changed_at DESC;
END;
$ LANGUAGE plpgsql;
\end{lstlisting}

This function provides a comprehensive view of how a parameter has evolved over time, including who made changes, which variants were affected, and the specific value modifications. The function uses JSON path operators to extract specific values from the JSONB store, implementing what Obe and Hsu \cite{obe2017postgresql} describe as the document extraction pattern. By including user and phase information in the result set, the function provides comprehensive context for each change, supporting both diagnostic and compliance requirements.

\subsection{Partitioning Implementation}
\label{subsec:partitioning-implementation}

To address the potential performance and management challenges associated with the growing change history table, a partitioning strategy was implemented based on the phase model that forms the foundation of the VMAP system. PostgreSQL's declarative partitioning capabilities provide an efficient mechanism for dividing the large change\_history table into smaller, more manageable segments based on logical boundaries \cite{obe2017postgresql}.

\subsubsection{Partition Design}
\label{subsubsec:partition-design}

The partitioning strategy leverages the phase-based organization of parameter data, creating individual partitions for each development phase to align the physical storage structure with the logical organization of the data. This approach offers several advantages over time-based or release-based partitioning alternatives.

Phase-based partitioning creates a direct correspondence between the database's physical organization and the domain's logical structure. According to Schwartz et al. \cite{schwartz2012high}, alignment between logical data organization and physical partitioning is an essential factor for successful partition design.

Queries filtering by phase (a predominant access pattern in the system) benefit from partition pruning, where PostgreSQL automatically eliminates irrelevant partitions from consideration. This pruning mechanism reduces I/O and improves query response time for phase-specific operations \cite{obe2017postgresql}.

Additionally, phase-based partitioning simplifies maintenance operations such as archiving or removing historical data for completed phases. When phases transition from active development to frozen status, their corresponding partitions can be managed accordingly without affecting ongoing development work.

The implementation uses PostgreSQL's LIST partitioning strategy, with phase\_id as the partition key:

\begin{lstlisting}[language=SQL, caption={Change History Table Partitioning Definition}, label={lst:change-history-partitioning}]
-- Create the partitioned change_history table
CREATE TABLE change_history (
    change_id BIGINT NOT NULL,
    user_id BIGINT REFERENCES users(user_id),
    ecu_id INTEGER,
    phase_id INTEGER, 
    entity_type VARCHAR(50) NOT NULL,
    entity_id BIGINT NOT NULL,
    change_type VARCHAR(50),
    old_values JSONB,
    new_values JSONB,
    changed_at TIMESTAMP WITH TIME ZONE DEFAULT CURRENT_TIMESTAMP,
    transaction_id BIGINT NOT NULL
) PARTITION BY LIST (phase_id);

-- Create default partition for NULL phase_id values 
-- (for system-level changes not tied to a specific phase)
CREATE TABLE change_history_default PARTITION OF change_history DEFAULT;
ALTER TABLE change_history_default ADD PRIMARY KEY (change_id);

-- Create indexes on the default partition
CREATE INDEX idx_ch_default_entity ON change_history_default(entity_type, entity_id, changed_at);
CREATE INDEX idx_ch_default_user ON change_history_default(user_id, changed_at);
CREATE INDEX idx_ch_default_ecu ON change_history_default(ecu_id, changed_at);
CREATE INDEX idx_ch_default_transaction ON change_history_default(transaction_id, changed_at);
CREATE INDEX idx_ch_default_changed_at ON change_history_default(changed_at);
\end{lstlisting}

The default partition handles system-level changes that are not associated with specific phases, ensuring all change records have an appropriate storage location even if they don't fit the defined partition scheme. This approach aligns with PostgreSQL best practices for handling exceptions to the primary partitioning scheme \cite{obe2017postgresql}.

\subsubsection{Automated Partition Management}
\label{subsubsec:automated-partition-management}

To ensure that appropriate partitions exist for each phase, an automated partition creation mechanism was implemented using database triggers. When a new phase is created, the system automatically creates a corresponding partition for change history records associated with that phase:

\begin{lstlisting}[language=SQL, caption={Automatic Partition Creation}, label={lst:auto-partition-creation}]
-- Procedure to safely create partitions by moving data first
CREATE OR REPLACE PROCEDURE create_change_history_partition_for_phase(p_phase_id INTEGER, p_phase_name VARCHAR)
LANGUAGE plpgsql
AS $$
DECLARE
    partition_name TEXT;
    partition_exists BOOLEAN;
    row_count INTEGER;
BEGIN
    partition_name := 'change_history_p' || p_phase_id;
    
    -- Check if partition already exists
    SELECT EXISTS (
        SELECT FROM pg_class c
        JOIN pg_namespace n ON n.oid = c.relnamespace
        WHERE c.relname = partition_name AND n.nspname = current_schema()
    ) INTO partition_exists;
    
    IF NOT partition_exists THEN
        -- Check if we have rows in the default partition with this phase_id
        EXECUTE 'SELECT COUNT(*) FROM change_history_default WHERE phase_id = $1'
        INTO row_count
        USING p_phase_id;
        
        IF row_count > 0 THEN
            -- Move the rows to a temporary table
            EXECUTE 'CREATE TEMPORARY TABLE temp_change_history AS 
                     SELECT * FROM change_history_default 
                     WHERE phase_id = $1'
            USING p_phase_id;
            
            -- Delete those rows from the default partition
            EXECUTE 'DELETE FROM change_history_default 
                     WHERE phase_id = $1'
            USING p_phase_id;
        END IF;
        
        -- Create the partition
        EXECUTE format(
            'CREATE TABLE %I PARTITION OF change_history FOR VALUES IN (%L)',
            partition_name, p_phase_id
        );
        
        -- Add primary key
        EXECUTE format(
            'ALTER TABLE %I ADD PRIMARY KEY (change_id)',
            partition_name
        );
        
        -- Add comment
        EXECUTE format(
            'COMMENT ON TABLE %I IS %L',
            partition_name, 'Change history for phase ' || p_phase_name
        );
        
        -- Add indexes
        EXECUTE format(
            'CREATE INDEX idx_%s_entity ON %I(entity_type, entity_id, changed_at)',
            substring(partition_name from 'change_history_(.*)'), partition_name
        );
        
        EXECUTE format(
            'CREATE INDEX idx_%s_user ON %I(user_id, changed_at)',
            substring(partition_name from 'change_history_(.*)'), partition_name
        );
        
        EXECUTE format(
            'CREATE INDEX idx_%s_ecu ON %I(ecu_id, changed_at)',
            substring(partition_name from 'change_history_(.*)'), partition_name
        );
        
        EXECUTE format(
            'CREATE INDEX idx_%s_transaction ON %I(transaction_id, changed_at)',
            substring(partition_name from 'change_history_(.*)'), partition_name
        );
        
        EXECUTE format(
            'CREATE INDEX idx_%s_changed_at ON %I(changed_at)',
            substring(partition_name from 'change_history_(.*)'), partition_name
        );
        
        -- If we moved rows, insert them back into the new partition
        IF row_count > 0 THEN
            EXECUTE 'INSERT INTO ' || partition_name || ' SELECT * FROM temp_change_history';
            
            -- Clean up the temporary table
            EXECUTE 'DROP TABLE temp_change_history';
        END IF;
        
        RAISE NOTICE 'Created partition % for phase % (ID %). Moved % rows.', 
                    partition_name, p_phase_name, p_phase_id, row_count;
    END IF;
END;
$$;

-- Function to trigger partition creation when a new phase is created
CREATE OR REPLACE FUNCTION create_phase_partition_trigger()
RETURNS TRIGGER
LANGUAGE plpgsql
AS $$
BEGIN
    CALL create_change_history_partition_for_phase(NEW.phase_id, NEW.name);
    RETURN NEW;
END;
$$;

-- Trigger to automatically create partitions for new phases
CREATE TRIGGER trig_create_phase_partition
AFTER INSERT ON release_phases
FOR EACH ROW
EXECUTE FUNCTION create_phase_partition_trigger();
\end{lstlisting}

This implementation follows the approach described by Schwartz et al. \cite{schwartz2012high} for maintaining partitioned tables in production environments. The procedure handles the complex task of safely creating new partitions, including migrating any existing records from the default partition to the newly created phase-specific partition. This migration step maintains data integrity during the partition creation process.

The partitioning implementation is particularly noteworthy for its attention to index creation. Each partition receives a complete set of indexes matching those on the default partition, ensuring consistent query performance across all partitions. According to Karwin \cite{karwin2010sql}, maintaining consistent indexing across partitions is essential for predictable query execution plans and optimal performance.

\subsubsection{Query Performance Implications}
\label{subsubsec:query-performance-implications}

The phase-based partitioning approach has significant performance implications for change history queries. For queries that include a phase\_id filter criterion—a dominant pattern in the VMAP system—PostgreSQL can use partition pruning to eliminate irrelevant partitions from consideration. Salzberg and Tsotras \cite{salzberg1999comparison} note that with effective partitioning, query performance can improve by an order of magnitude for large temporal datasets.

This performance benefit is particularly valuable for the most common access patterns in the system. When comparing parameters between phases, queries can focus exclusively on the relevant phase partitions. When auditing a user's activity within a specific phase, queries can be efficiently directed to the appropriate partition. When retrieving the modification history for a specific entity within a phase, partition pruning significantly reduces the search space.

Importantly, the partitioning scheme is transparent to application code, requiring no modifications to existing queries. The PostgreSQL query planner automatically applies partition pruning based on the WHERE clause predicates, allowing the application to benefit from partitioning without explicit partition selection in SQL statements.

\subsubsection{Archiving Strategy}
\label{subsubsec:archiving-strategy}

The phase-based partitioning strategy provides a natural foundation for long-term data archiving. As phases transition from active development to frozen status, their change history records become less frequently accessed but must be preserved for regulatory compliance and occasional historical analysis.

The partitioning implementation supports a future archiving strategy where frozen phase partitions could be compressed using PostgreSQL's table compression options to reduce storage requirements. According to Bhattacherjee et al. \cite{bhattacherjee2015principles}, efficient archiving strategies are essential for managing the growing storage requirements of versioned datasets.

Frozen phase partitions could be moved to alternative tablespaces, relocating older partitions to lower-cost storage media while maintaining accessibility. For very old phases that have progressed to production, partitions could be detached from the main table and accessed only when specifically required. These archiving capabilities align with the recommendations of Al-Kateb et al. \cite{al2013temporal} for managing historical data in temporal database systems, where different storage tiers can be leveraged based on data age and access frequency. Snodgrass \cite{snodgrass1999developing} suggests that this tiered approach to historical data management provides an optimal balance between accessibility and resource utilization.

At defined archive thresholds, typically when a release including multiple phases transitions to production status, historical data can be compressed or moved while preserving selective access capabilities. This approach ensures that historical reference data remains available for compliance purposes while optimizing system performance for active development phases.


\section{Integration Implementation}
\label{sec:integration-implementation}

The integration implementation connects VMAP with external enterprise systems providing parameter definitions and vehicle configuration data. Two primary integration points were implemented: synchronization with the Parameter Definition Database and communication with the Vehicle Configuration Database.

\subsection{Parameter Definition Database Synchronization}
\label{subsec:pdd-synchronization}

Synchronization with the Parameter Definition Database (PDD) was implemented through a structured process that maintains parameter definitions across different release phases. The database structure for tracking synchronization includes:

\begin{lstlisting}[language=SQL, caption={Synchronization Tracking Implementation}, label={lst:sync-tracking}]
CREATE TABLE pdd_sync_history (
    sync_id INTEGER PRIMARY KEY,
    ecu_id INTEGER,
    phase_id INTEGER,
    database_name VARCHAR(255),
    database_label VARCHAR(255),
    sync_date TIMESTAMP WITH TIME ZONE DEFAULT CURRENT_TIMESTAMP,
    status VARCHAR(50) NOT NULL,
    modules_count INTEGER DEFAULT 0,
    pids_count INTEGER DEFAULT 0,
    parameters_count INTEGER DEFAULT 0,
    executed_by BIGINT REFERENCES users(user_id),
    transaction_id BIGINT,
    entity_changes TEXT,
    FOREIGN KEY (ecu_id, phase_id) REFERENCES ecu_phases(ecu_id, phase_id)
);
\end{lstlisting}

This infrastructure implements the external system registry pattern described by Hohpe and Woolf \cite{hohpe2002enterprise}, tracking synchronization operations with external database systems. The detailed history supports both operational monitoring and troubleshooting of integration issues.

The parameter loading implementation uses a bulk loading approach for efficiency:

\begin{lstlisting}[language=SQL, caption={Bulk Parameter Loading Implementation}, label={lst:bulk-loading}]
CREATE OR REPLACE FUNCTION load_parameters_bulk(
    p_parameters JSONB,
    p_ecu_id INTEGER,
    p_phase_id INTEGER,
    p_user_id BIGINT
)
RETURNS INTEGER AS $
DECLARE
    v_count INTEGER;
BEGIN
    -- Create temporary table for bulk loading
    CREATE TEMPORARY TABLE temp_parameters (
        parameter_id BIGINT,
        pid_id BIGINT,
        name VARCHAR(255),
        parameter_name VARCHAR(255),
        type_id INTEGER,
        array_definition VARCHAR(50),
        position INTEGER,
        factor DECIMAL,
        unit VARCHAR(50),
        bias_offset DECIMAL,
        external_id INTEGER
    ) ON COMMIT DROP;
    
    -- Load data from JSON into temporary table
    INSERT INTO temp_parameters
    SELECT 
        (elem->>'parameter_id')::BIGINT,
        (elem->>'pid_id')::BIGINT,
        elem->>'name',
        elem->>'parameter_name',
        (elem->>'type_id')::INTEGER,
        elem->>'array_definition',
        (elem->>'position')::INTEGER,
        (elem->>'factor')::DECIMAL,
        elem->>'unit',
        (elem->>'bias_offset')::DECIMAL,
        (elem->>'external_id')::INTEGER
    FROM jsonb_array_elements(p_parameters) AS elem;
    
    -- Bulk insert into parameters table
    INSERT INTO parameters (
        parameter_id, pid_id, ecu_id, phase_id, 
        name, parameter_name, type_id, array_definition,
        position, factor, unit, bias_offset, external_id,
        created_by
    )
    SELECT 
        tp.parameter_id, tp.pid_id, p_ecu_id, p_phase_id,
        tp.name, tp.parameter_name, tp.type_id, tp.array_definition,
        tp.position, tp.factor, tp.unit, tp.bias_offset, tp.external_id,
        p_user_id
    FROM temp_parameters tp
    ON CONFLICT (parameter_id) DO UPDATE SET
        name = EXCLUDED.name,
        parameter_name = EXCLUDED.parameter_name,
        type_id = EXCLUDED.type_id,
        array_definition = EXCLUDED.array_definition,
        position = EXCLUDED.position,
        factor = EXCLUDED.factor,
        unit = EXCLUDED.unit,
        bias_offset = EXCLUDED.bias_offset;
    
    GET DIAGNOSTICS v_count = ROW_COUNT;
    RETURN v_count;
END;
$ LANGUAGE plpgsql;
\end{lstlisting}

This bulk loading approach uses temporary tables and JSON parsing to efficiently process large volumes of parameter data, implementing the bulk transfer pattern described by Hohpe and Woolf \cite{hohpe2002enterprise}. The use of PostgreSQL's ON CONFLICT clause implements an efficient upsert operation, enabling both insertion of new parameters and updating of existing ones in a single operation.

\subsection{Vehicle Configuration Database Integration}
\label{subsec:vcd-integration}

The Vehicle Configuration Database (VCD) integration enables validation of variant code rules and supports parameter file generation for specific vehicle configurations. The database structure includes tables for vehicles and their associated configuration codes:

\begin{lstlisting}[language=SQL, caption={Vehicle Configuration Tables}, label={lst:vehicle-configuration}]
CREATE TABLE vcd_vehicles (
    vehicle_id INTEGER PRIMARY KEY,
    vcd_vehicle_id VARCHAR(100) UNIQUE NOT NULL,
    name VARCHAR(255) NOT NULL,
    description TEXT,
    is_active BOOLEAN DEFAULT true,
    last_sync_at TIMESTAMP WITHOUT TIME ZONE,
    created_at TIMESTAMP WITHOUT TIME ZONE DEFAULT CURRENT_TIMESTAMP
);

CREATE TABLE vehicle_codes (
    code_id INTEGER PRIMARY KEY,
    vcd_code_id VARCHAR(100) UNIQUE,
    code VARCHAR(50) NOT NULL,
    vehicle_type VARCHAR(100),
    description TEXT,
    is_active BOOLEAN DEFAULT true,
    created_at TIMESTAMP WITHOUT TIME ZONE DEFAULT CURRENT_TIMESTAMP
);

CREATE TABLE vehicle_code_mapping (
    vehicle_id INTEGER REFERENCES vcd_vehicles(vehicle_id) ON DELETE CASCADE,
    code_id INTEGER REFERENCES vehicle_codes(code_id) ON DELETE CASCADE,
    created_at TIMESTAMP WITHOUT TIME ZONE DEFAULT CURRENT_TIMESTAMP,
    PRIMARY KEY (vehicle_id, code_id)
);
\end{lstlisting}

This structure separates vehicles from codes with a mapping table, implementing the many-to-many relationship pattern described by Elmasri and Navathe \cite{elmasri2015fundamentals}. External identifiers (\texttt{vcd\_vehicle\_id}, \texttt{vcd\_code\_id}) link to the Vehicle Configuration Database system, implementing the integration reference pattern described by Hohpe and Woolf \cite{hohpe2002enterprise}.

A critical component of this integration is the code rule evaluation engine, which determines when variants apply to specific vehicles. The implementation uses a postfix expression evaluator to process boolean expressions efficiently:

\begin{lstlisting}[language=SQL, caption={Code Rule Evaluation Implementation}, label={lst:code-rule-evaluation}]
CREATE OR REPLACE FUNCTION evaluate_code_rule(
    p_rule TEXT,
    p_vehicle_id INTEGER
)
RETURNS BOOLEAN AS $
DECLARE
    v_tokens TEXT[];
    v_stack BOOLEAN[] DEFAULT '{}';
    v_token TEXT;
    v_operand1 BOOLEAN;
    v_operand2 BOOLEAN;
    v_code_exists BOOLEAN;
    v_i INTEGER;
BEGIN
    -- If rule is empty, it applies to all vehicles
    IF p_rule IS NULL OR p_rule = '' THEN
        RETURN TRUE;
    END IF;
    
    -- Tokenize the rule (assuming rule is in postfix notation)
    v_tokens := string_to_array(p_rule, ' ');
    
    -- Process each token
    FOR v_i IN 1..array_length(v_tokens, 1) LOOP
        v_token := v_tokens[v_i];
        
        -- Check if token is an operator
        IF v_token = 'AND' THEN
            IF array_length(v_stack, 1) < 2 THEN
                RAISE EXCEPTION 'Invalid rule expression: insufficient operands for AND';
            END IF;
            v_operand2 := v_stack[array_length(v_stack, 1)];
            v_operand1 := v_stack[array_length(v_stack, 1) - 1];
            v_stack := v_stack[1:array_length(v_stack, 1) - 2];
            v_stack := array_append(v_stack, v_operand1 AND v_operand2);
            
        ELSIF v_token = 'OR' THEN
            IF array_length(v_stack, 1) < 2 THEN
                RAISE EXCEPTION 'Invalid rule expression: insufficient operands for OR';
            END IF;
            v_operand2 := v_stack[array_length(v_stack, 1)];
            v_operand1 := v_stack[array_length(v_stack, 1) - 1];
            v_stack := v_stack[1:array_length(v_stack, 1) - 2];
            v_stack := array_append(v_stack, v_operand1 OR v_operand2);
            
        ELSIF v_token = 'NOT' THEN
            IF array_length(v_stack, 1) < 1 THEN
                RAISE EXCEPTION 'Invalid rule expression: insufficient operands for NOT';
            END IF;
            v_operand1 := v_stack[array_length(v_stack, 1)];
            v_stack := v_stack[1:array_length(v_stack, 1) - 1];
            v_stack := array_append(v_stack, NOT v_operand1);
            
        ELSE
            -- Token is a vehicle code - check if it exists for this vehicle
            SELECT EXISTS (
                SELECT 1
                FROM vehicle_code_mapping vcm
                JOIN vehicle_codes vc ON vcm.code_id = vc.code_id
                WHERE vcm.vehicle_id = p_vehicle_id
                AND vc.code = v_token
            ) INTO v_code_exists;
            
            v_stack := array_append(v_stack, v_code_exists);
        END IF;
    END LOOP;
    
    -- Result should be a single boolean value on the stack
    IF array_length(v_stack, 1) != 1 THEN
        RAISE EXCEPTION 'Invalid rule expression: did not evaluate to a single result';
    END IF;
    
    RETURN v_stack[1];
END;
$ LANGUAGE plpgsql;
\end{lstlisting}

This evaluation function implements a stack-based interpreter for boolean expressions, efficiently determining whether a variant applies to a specific vehicle based on its configuration codes. The implementation follows the interpreter pattern described by Fowler \cite{fowler2003patterns}, translating a domain-specific language (boolean expressions) into executable operations. The use of PostgreSQL's array operations enables efficient stack manipulation without requiring complex data structures.

\subsection{Parameter File Generation}
\label{subsec:parameter-file-generation}

The parameter file generation capability represents a core integration point between the version control system and vehicle testing infrastructure. The database structure includes records for generated parameter files:

\begin{lstlisting}[language=SQL, caption={Parameter File Records}, label={lst:par-file-records}]
CREATE TABLE par_files (
    par_file_id BIGINT PRIMARY KEY,
    vehicle_id INTEGER REFERENCES vcd_vehicles(vehicle_id),
    ecu_id INTEGER,
    phase_id INTEGER,
    generated_at TIMESTAMP WITH TIME ZONE DEFAULT CURRENT_TIMESTAMP,
    generated_by BIGINT REFERENCES users(user_id),
    description TEXT,
    FOREIGN KEY (ecu_id, phase_id) REFERENCES ecu_phases(ecu_id, phase_id)
);
\end{lstlisting}

This table tracks the generation of parameter files, recording metadata about when they were created, which vehicle configuration they target, and which user generated them. This tracking mechanism supports both operational auditing and diagnostic capabilities when issues arise with parameter files.

The parameter file generation process implements a multi-step algorithm:

\begin{itemize}
  \item Retrieve vehicle configuration codes from the Vehicle Configuration Database
  \item Evaluate variant code rules against the vehicle configuration to determine applicable variants
  \item Resolve parameter values using the variant resolution process
  \item Format the resolved values according to the ECU-specific parameter file format
  \item Record the parameter file generation in the database for traceability
\end{itemize}

According to Staron \cite{staron2021automotive}, this integration point is critical in automotive software development, transforming abstract parameter configurations into testable implementations that can be validated on actual hardware.
